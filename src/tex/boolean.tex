\documentclass[11pt,oneside]{article}	%use"amsart"insteadof"article"forAMSLaTeXformat
\usepackage{geometry}		%Seegeometry.pdftolearnthelayoutoptions.Therearelots.
\geometry{letterpaper}		%...ora4paperora5paperor...
%\geometry{landscape}		%Activateforforrotatedpagegeometry
%\usepackage[parfill]{parskip}		%Activatetobeginparagraphswithanemptylineratherthananindent
\usepackage{graphicx}				%Usepdf,png,jpg,orepsßwithpdflatex;useepsinDVImode
								%TeXwillautomaticallyconverteps-->pdfinpdflatex		
\usepackage{amssymb}
\usepackage{hyperref}

%----macros begin---------------------------------------------------------------
\usepackage{color}
\usepackage{amsthm}

\def\conv{\mbox{\textrm{conv}\,}}
\def\aff{\mbox{\textrm{aff}\,}}
\def\E{\mathbb{E}}
\def\R{\mathbb{R}}
\def\Z{\mathbb{Z}}
\def\tex{\TeX}
\def\latex{\LaTeX}
\def\v#1{{\bf #1}}
\def\p#1{{\bf #1}}
\def\T#1{{\bf #1}}

\def\vet#1{{\left(\begin{array}{cccccccccccccccccccc}#1\end{array}\right)}}
\def\mat#1{{\left(\begin{array}{cccccccccccccccccccc}#1\end{array}\right)}}

\def\lin{\mbox{\rm lin}\,}
\def\aff{\mbox{\rm aff}\,}
\def\pos{\mbox{\rm pos}\,}
\def\cone{\mbox{\rm cone}\,}
\def\conv{\mbox{\rm conv}\,}
\newcommand{\homog}[0]{\mbox{\rm homog}\,}
\newcommand{\relint}[0]{\mbox{\rm relint}\,}

%----macros end-----------------------------------------------------------------

\title{Boolean Chains: set operations with topological chains
\footnote{This document is part of the \emph{Linear Algebraic Representation with CoChains} (LAR-CC) framework~\cite{cclar-proj:2013:00}. \today}
}
\author{TheAuthor}
%\date{}							%Activatetodisplayagivendateornodate

\begin{document}
\maketitle
%\nonstopmode

%----macros end-----------------------------------------------------------------
%>>>>>>>>>>>>>>>>>>>>>>>>>>>>>>>>>>>>>>>>>>>>>>>>>>>>>>>>>>>>>>>>>>>>>>>>>>>>>>>
\begin{abstract}
Boolean operations are a major addition to every geometric package. Union, intersection, difference and complementation of decomposed spaces are discussed and implemented in this module by making use of the Linear Algebraic Representation (LAR) introduced in~\cite{Dicarlo:2014:TNL:2543138.2543294}. First, the two finite decompositions are merged, by merging their vertices (0-cells of support spaces); then a Delaunay complex of set union of their boundary vertices is computed, and the shared $d$-chain is extracted and split, according to the cellular structure of the input $d$-chains. The results of a Boolean operation are finally computed by sum, product or difference of the (binary) coordinate representation of the (split) argument chains, by using the novel chain-basis resulted from such boundary-based splitting. Differently from the totality of algorithms known to the author, neither search nor traversal of some (complicated) data structure is performed by this algorithm. 
\end{abstract}

\tableofcontents
%<<<<<<<<<<<<<<<<<<<<<<<<<<<<<<<<<<<<<<<<<<<<<<<<<<<<<<<<<<<<<<<<<<<<<<<<<<<<<<<
%>>>>>>>>>>>>>>>>>>>>>>>>>>>>>>>>>>>>>>>>>>>>>>>>>>>>>>>>>>>>>>>>>>>>>>>>>>>>>>>
\section{Introduction}

\bibliographystyle{amsalpha}
\bibliography{boolean}

\end{document}
